% !TEX root = ./paper/paper.tex
%% This is an abbreviated template from http://www.sigplan.org/Resources/Author/.

\documentclass[acmsmall,review,nonacm]{acmart}
\begin{document}

%%
%% The "title" command has an optional parameter,
%% allowing the author to define a "short title" to be used in page headers.
\title{The Name of the Title is Hope}

%%
%% The "author" command and its associated commands are used to define
%% the authors and their affiliations.
%% Of note is the shared affiliation of the first two authors, and the
%% "authornote" and "authornotemark" commands
%% used to denote shared contribution to the research.
\author{Aka Sai Lalith Kumar}
\email{sailalithkumar.aka@colorado.edu}
\affiliation{%
  \institution{University of Colorado Boulder}
  \country{USA}
}


%%
%% The abstract is a short summary of the work to be presented in the
%% article.
\begin{abstract}
We have a lot of theoretically rigorous over-approximation logics, which is to be expected given the rich treatment they have been enjoying for the last 5 decades. 
Under-approximation is the other side of the coin where the potential lies for development. 
An area which is nascent would be the application of Under-approximation logics to security models. 
This takes us from the realm of showing a protocol is secure to finding an attack within the protocol which is guarnteed to exist.
This is a value add, and something the security industry has already adopted though without much formal methods through exploitability analysis. 
In this work, we aim to show with a theortical guarntee that the current core logics which support them are what they promise to be.
They are sound, and they do have the capabilities to detect iterative, multistage attacks where adversaries interact with the loop.
This effectively bridges the gap from current knowledge about exploitability and increases further confidence in the dissemination of the technique.
\end{abstract}

%%
%% This command processes the author and affiliation and title
%% information and builds the first part of the formatted document.
\maketitle

\section{Introduction}

\section{Overview}

\section{Semantics}

\section{Soundness}

\section{Bounded Completness}

\section{Evaluation}

\section{Related Work}

\section{Conclusion}

%%
%% The acknowledgments section is defined using the "acks" environment
%% (and NOT an unnumbered section). This ensures the proper
%% identification of the section in the article metadata, and the
%% consistent spelling of the heading.
\begin{acks}
Bohr-Yuh Evan Chang, Kirby Linvill and Gowtham Kaki.

Dakota Brayan

Fabio Somenzi


\end{acks}

%%
%% The next two lines define the bibliography style to be used, and
%% the bibliography file.
\bibliographystyle{ACM-Reference-Format}
\bibliography{paper}
\end{document}
